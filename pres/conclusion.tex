\begin{frame}
  \frametitle{Conclusion}
        These results demonstrate that by carefully selecting \deploy 
        parameters, we are able to \textbf{effectively automate deployment}
        of reactors and supporting facilities to simulate
        constant and linearly increasing power demand transition scenarios
        for EG01-23, EG01-24, EG01-29, and EG01-30 with minimal 
        power undersupply. 
        \vspace{1em}
        \\
        Not completely eliminating undersupply and under capacity of 
        commodities in the simulation is expected 
        since without time series data 
        at the beginning of the simulation, \deploy takes a few 
        time steps to collect time series data about power demand 
        to predict and start deploying reactor and supporting 
        fuel cycle facilities. 
        
\end{frame}

\begin{frame}
  \frametitle{Future Work}
  \deploy can be used to conduct nuclear fuel cycle \textbf{sensitivity studies}. 
  One of the key issues facing nuclear fuel cycle transition scenario 
  simulations is the presence 
  of idle reactor capacity due to the lack of Pu to fabricate advanced fuels 
  in the simulation. 
  Previously, to conduct sensitivity analysis,  the user would have to manually 
  calculate the deployment scheme for every change in input parameter to avoid 
  idle capacity. 
\end{frame}