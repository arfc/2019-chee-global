\documentclass{anstrans}
%%%%%%%%%%%%%%%%%%%%%%%%%%%%%%%%%%%
\title{Demonstration of Demand Driven Deployment Capabilities in \Cyclus}
\author{Gwendolyn J. Chee$^1$, Jin Whan Bae$^1$, Robert R. Flanagan$^2$, Roberto E. Fairhurst$^1$ and Kathryn D. Huff$^1$}

\institute{
$^1$Dept. of Nuclear, Plasma and Radiological Engineering, University of Illinois at Urbana-Champaign \\
$^2$Nuclear Engineering Program, University of South Carolina \\

gchee2@illinois.edu
}

%%%% packages and definitions (optional)
\usepackage{graphicx} % allows inclusion of graphics
\usepackage{booktabs} % nice rules (thick lines) for tables
\usepackage{microtype} % improves typography for PDF
\usepackage{xspace}
\usepackage{tabularx}
\usepackage{subcaption}
\usepackage{enumitem}
\usepackage{placeins}
\usepackage[acronym,toc]{glossaries}
\include{acros}
\makeglossaries
\newcommand{\SN}{S$_N$}
\renewcommand{\vec}[1]{\bm{#1}} %vector is bold italic
\newcommand{\vd}{\bm{\cdot}} % slightly bold vector dot
\newcommand{\grad}{\vec{\nabla}} % gradient
\newcommand{\ud}{\mathop{}\!\mathrm{d}} % upright derivative symbol
\newcommand{\Cyclus}{\textsc{Cyclus}\xspace}%
\newcommand{\Cycamore}{\textsc{Cycamore}\xspace}%
\newcolumntype{c}{>{\hsize=.56\hsize}X}
\newcolumntype{b}{>{\hsize=.7\hsize}X}
\newcolumntype{s}{>{\hsize=.74\hsize}X}
\newcolumntype{f}{>{\hsize=.1\hsize}X}
\newcolumntype{a}{>{\hsize=.45\hsize}X}
\usepackage{titlesec}
\titleformat*{\subsection}{\normalfont}

\begin{document}
%%%%%%%%%%%%%%%%%%%%%%%%%%%%%%%%%%%%%%%%%%%%%%%%%%%%%%%%%%%%%%%%%%%%
\section{Introduction}
- Description of the current state of nuclear fuel cycle simulators
(lacking automatic deployment) 

- Why there is a need for a functionality like d3ploy 

- The objectives of this paper: explain the capabilities of d3ploy, 
demonstrate how d3ploy is able to meet the primary objective of 
minimizing undersupply and undercapacity, demonstrate d3ploy's use 
in a transition scenario.  

\section{Background}
- What is \Cyclus 

- What is d3ploy in relation to \Cyclus 

\section{Method}
- describe the capabilities of d3ploy and why they were developed
this way 

- explain the performance tests and what they are meant to demonstrate
and why each of them was performed. 

- explain why a supply/capacity buffer was introduced. 

\section{Results}
- Show the results for each performance test with and without 
supply and capacity buffers. (Scenario 3 and 7 in particular, 
these are the scenarios with a refueling reactor)

- Show the results for one transition scenario with and without
d3ploy. Explain how the use of d3ploy fills the gaps in the 
capability of \Cyclus for modeling transition scenarios. 
[Include a figure of the commodity flow in the transition scenario]  

- highlight the importance of a supply/capacity buffer for 
ensuring that there is enough of each commodity when there is 
a spike due to refueling of a reactor etc. 

\section{Conclusion}
- d3ploy rules! 

\section{Acknowledgements}

%%%%%%%%%%%%%%%%%%%%%%%%%%%%%%%%%%%%%%%%%%%%%%%%%%%%%%%%%%%%%%%%%%%%
\bibliographystyle{ans}
\bibliography{bibliography}
\end{document}

