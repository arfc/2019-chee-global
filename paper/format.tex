\documentclass[11pt,letterpaper]{article}
\usepackage[utf8]{inputenc}
\usepackage{caption} % for table captions
\usepackage{amsmath} % for multi-line equations and piecewises
\DeclareMathOperator{\sign}{sign}
\usepackage{graphicx}
\usepackage{relsize}
%\usepackage{textcomp}
\usepackage{xspace}
\usepackage{verbatim} % for block comments
%\usepackage{subfig} % for subfigures
\usepackage{enumitem} % for a) b) c) lists
\newcommand{\Cyclus}{\textsc{Cyclus}\xspace}%
\newcommand{\Cycamore}{\textsc{Cycamore}\xspace}%
\newcommand{\deploy}{\texttt{d3ploy}\xspace}%
\usepackage{tabularx}
\usepackage{color}
\usepackage[acronym,toc]{glossaries}
\include{acros}
\definecolor{bg}{rgb}{0.95,0.95,0.95}
\newcolumntype{b}{X}
\newcolumntype{f}{>{\hsize=.15\hsize}X}
\newcolumntype{s}{>{\hsize=.5\hsize}X}
\newcolumntype{m}{>{\hsize=.75\hsize}X}
\newcolumntype{r}{>{\hsize=1.1\hsize}X}
\usepackage{titling}
\usepackage[hang,flushmargin]{footmisc}
\renewcommand*\footnoterule{}
\usepackage{tikz}


\usetikzlibrary{shapes.geometric,arrows}
\tikzstyle{process} = [rectangle, rounded corners, 
minimum width=1cm, minimum height=1cm,text centered, draw=black, 
fill=blue!30]
\tikzstyle{arrow} = [thick,->,>=stealth]


\graphicspath{{figures/}}
\title{DDCA GLOBAL}
\author{Gwendolyn J. Chee, Roberto E. Fairhurst, 
\\ \vspace{0.5em} Robert R. Flanagan, Kathryn D. Huff}


\begin{document}
	\begin{titlepage}
	\maketitle
	\thispagestyle{empty}
	\end{titlepage}

%----------------------------------------------------------------%
\section{Introduction}
\gls{NFC} simulation scenarios are constrained objective functions. 
The objectives are systemic demands such as "1\% power growth", 
while an example of a constraint is the availability of new nuclear 
technology. 
To add ease in setting up nuclear fuel cycle simulations, \gls{NFC}
simulators should bring demand responsive deployment decisions into 
the dynamics of the simulation logic \cite{huff_current_2017}. 
While automated power production deployment is common in most fuel 
cycle simulators, automated deployment of supportive fuel cycle 
facilities is non-existent. 

*- The capabilities in current fuel cycle sims (ORION, etc)

Instead, the user must detail the deployment timeline of all 
supporting facilities or have infinite capacity support facilities. 
Thus, a next generation \gls{NFC} simulator should predictively and 
automatically deploy fuel cycle facilities to meet user defined 
power demand. 
This paper discusses demonstration of a new solution, 
\texttt{d3ploy}, enabling demand driven deployment in \Cyclus. 

\Cyclus is an agent-based nuclear fuel cycle simulation framework 
\cite{huff_fundamental_2016}. 
Each entity (i.e. Region, Institution, or Facility) in the fuel 
cycle is modeled as an agent. 
Institution agents
are responsible for deploying and decommissioning facility agents 
and can represent a legal operating organization such as a 
utility, government, etc \cite{huff_fundamental_2016}. 

The Demand-Driven \Cycamore Archetypes project (NEUP-FY16-10512) 
aims to develop \Cyclus's demand-driven deployment capabilities. 
This capability is developed in the form of a \Cyclus Institution
agent that deploys facilities to meet the front-end and back-end 
fuel cycle demands based on a user-defined commodity demand. 
Its goal is to meet supply for any commodity while minimizing 
undersupply.
This demand-driven deployment capability is referred to as 
\deploy. 

In this paper, we will explain the capabilities of \deploy, 
demonstrate how \deploy can be used in various simulations (flat 
power demand, increasing power demand, sinusoidal power demand)
to meet the primary objective of minimizing undersupply of all 
commodities in the simulations, and finally give  . 


- The objectives of this paper: explain the capabilities of d3ploy, 
demonstrate how d3ploy is able to meet the primary objective of 
minimizing undersupply and undercapacity, demonstrate d3ploy's use 
in a transition scenario.  


\section{D3ploy capabilities}
- describe the capabilities of d3ploy and why they were developed
this way 

At each time step, \deploy predicts demand and supply of each 
commodity for the following time step.
Then, \deploy deploys facilities to meet predicted demand. 
\deploy's primary objective is to minimize the number of time 
steps of undersupply of any commodity. 

Within \deploy, there are two institutions: 
\texttt{DemandDrivenDeploymentInst} and \texttt{SupplyDrivenDeploymentInst}. 
The prior is used for the front-end of the fuel cycle and the latter is used 
for the back-end. 
Front-end facilities includes any facility that comes before the reactor, 
including the reactor such as a fuel fabrication facility etc.  
Back-end facilities includes any facility that comes after the reactor, 
such as a reprocessing facility etc. 
The reason for this separation is to let facilities have the choice 
to demand for supply or demand for capacity. 
For example, in the front end facilities, the reactor has a demand for 
fuel, using \texttt{DemandDrivenDeploymentInst}, it triggers the fuel 
fabrication facility to deploy facilities to create supply to meet 
the demand. 
Whereas, for the back end facilities, the reactor generates spent fuel 
, there is a demand for waste repository facility to accept the 
spent fuel, using \texttt{SupplyDrivenDeploymentInst}, it triggers the
deployment of a waste repository to create a capacity for spent fuel 
to meet the supply. 

\subsection{Prediction Algorithms}
Three interchangeable algorithm types govern demand and supply 
predictions: non-optimizing, deterministic optimizing and stochastic
optimizing. 

There are three methods implemented for the non-optimizing model: 
Autoregressive moving average (ARMA), autoregressive 
conditional heteroskedasticity (ARCH).
There are four methods implemented for the deterministic optimizing model: 
Polynomial fit regression, simple exponential smoothing,  
triple exponential smoothing (holt-winters) and fast fourier 
transform (fft). 
There is one method implemented for stochastic optimizing model: 
stepwise seasonal.  

\subsection{User defined input variables}
The user is able to input specific variables to customize their
simulation. 
Descriptions of each input variable can be found in the 
readme of the repository [https://github.com/arfc/d3ploy]. 
*- explain each system parameter that will be varied in the results.
System parameters include: supply/capacity buffer, varying calc 
methods for each commodity.   

Essentially, the user must define the facilities they want the 
institution to control and their corresponding capacities. 
The user must also define the driving commodity, its demand 
equation and what method they want the institution to predict 
demand and supply with. 

They also have the option to give an `equation` that governs
preference for that facility compared to other facilities that 
provide the same commodity. 
The user also has an option to constrain deployment of a facility 
until there is a accumulation of a specific commodity.   
The user also has an option to provide a supply or capacity buffer 
for each commodity, which is an amount above demand/supply the user 
wants the supply/capacity to meet. 

\section{Methods for demonstration of d3ploy capabilities}
*- explain the performance tests and what they are meant to 
demonstrate and why each of them was performed. 

Various simulations are set up to demonstrate \deploy's 
capabilities. 
A delicate balance between the various system parameters must be 
met for each type of simulation to ideally meet the goal of 
minimizing undersupply and undercapacity for the various 
commodities. 

\section{Results}
- Show the results for each performance test with and without 
supply and capacity buffers. (Scenario 3 and 7 in particular, 
these are the scenarios with a refueling reactor)

- Show the results for one transition scenario with and without
d3ploy. Explain how the use of d3ploy fills the gaps in the 
capability of \Cyclus for modeling transition scenarios. 
[Include a figure of the commodity flow in the transition scenario]  

- highlight the importance of a supply/capacity buffer for 
ensuring that there is enough of each commodity when there is 
a spike due to refueling of a reactor etc. 

\section{Conclusion}
- d3ploy rules! 

\section{Acknowledgements}

%----------------------------------------------------------------%

\pagebreak 
\bibliographystyle{plain}
\bibliography{bibliography}

\end{document}


