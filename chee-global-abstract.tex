\documentclass{anstrans}
%%%%%%%%%%%%%%%%%%%%%%%%%%%%%%%%%%%
\title{Demand Driven Deployment Capabilities in \Cyclus}
\author{Gwendolyn J. Chee, Jin Whan Bae, Robert R. Flanagan, Roberto Fairhurst and Kathryn D. Huff}

\institute{
Dept. of Nuclear, Plasma and Radiological Engineering, University of Illinois at Urbana-Champaign \\
gchee2@illinois.edu
}

%%%% packages and definitions (optional)
\usepackage{graphicx} % allows inclusion of graphics
\usepackage{booktabs} % nice rules (thick lines) for tables
\usepackage{microtype} % improves typography for PDF
\usepackage{xspace}
\usepackage{tabularx}
\usepackage{subcaption}
\usepackage{enumitem}
\usepackage{placeins}
\usepackage[acronym,toc]{glossaries}
\include{acros}
\makeglossaries
\newcommand{\SN}{S$_N$}
\renewcommand{\vec}[1]{\bm{#1}} %vector is bold italic
\newcommand{\vd}{\bm{\cdot}} % slightly bold vector dot
\newcommand{\grad}{\vec{\nabla}} % gradient
\newcommand{\ud}{\mathop{}\!\mathrm{d}} % upright derivative symbol
\newcommand{\Cyclus}{\textsc{Cyclus}\xspace}%
\newcommand{\Cycamore}{\textsc{Cycamore}\xspace}%
\newcolumntype{c}{>{\hsize=.56\hsize}X}
\newcolumntype{b}{>{\hsize=.7\hsize}X}
\newcolumntype{s}{>{\hsize=.74\hsize}X}
\newcolumntype{f}{>{\hsize=.1\hsize}X}
\newcolumntype{a}{>{\hsize=.45\hsize}X}
\usepackage{titlesec}
\titleformat*{\subsection}{\normalfont}

\begin{document}
%%%%%%%%%%%%%%%%%%%%%%%%%%%%%%%%%%%%%%%%%%%%%%%%%%%%%%%%%%%%%%%%%%%%%%%%%%%%%%%%
\section{Abstract}
\gls{NFC} simulation scenarios are constrained objective functions. 
The objectives are systemic demands such as "1\% power growth", while the 
constraints are availability of new nuclear technology. 
To effectively simulate a nuclear fuel cycle, \gls{NFC} simulators 
must bring demand responsive deployment decisions into the dynamics of the 
simulation logic \cite{huff_current_2017}. 
Thus, a \gls{NFC} simulator should have the capability to deploy 
supporting fuel cycle facilities to meet a user-defined commodity demand. 
While automated power production deployment is common in most fuel cycle simulators, automated deployment of supportive fuel cycle 
facilities is non-existent. 
Instead, the user must detail the deployment timeline of all supporting 
facilities or have infinite capacity support facilities. This shortcoming 
exists in the fuel cycle simulator, \Cyclus. 

\Cyclus is an agent-based nuclear fuel cycle simulation framework 
\cite{huff_fundamental_2016}. 
Each entity (i.e. Region, Institution, or Facility) in the fuel cycle is modeled 
as an agent. 
Institution agents
are responsible for the deployment and decommissioning of facility agents and
The institution agent represents a legal operating organization such as a 
utility, government, etc \cite{huff_fundamental_2016}. 

The Demand-Driven \Cycamore Archetype project (NEUP-FY16-10512) aims to 
develop \Cyclus's demand-driven deployment capabilities. 
This capability is developed in the form of a \Cyclus Institution agent that 
deploys facilities to meet the front and back-end demands of the fuel cycle 
based on a user-defined commodity demand. 
Its goal is to minimize the time where demand exceeds supply for any commodity.
The demand-driven deployment capabilities is referred to as d3ploy. 

At each time step, demand and supply for each commodity is predicted for the 
following time step. Based on the prediction, facilities will be deployed to 
meet predicted demand. 
The demand and supply predictions are governed by four types of algorithms: 
non-optimizing, time series forecasting, deterministic optimizing and machine 
learning. 
The choice of which prediction algorithm to use is a user-input. 

The prediction algorithms are compared using numerical experiments. 
The numerical experiments are in the form of fuel cycle scenarios where the
demand driving commodity, its demand curve and the combination of facilities 
in the scenario are varied. 
The results of the numerical experiments are compared based on number of time
steps where demand exceeds supply, residuals and chi goodness of fit test. 
The non-optimizing methods were effective for predicting the driving commodity.
Whereas, the time series forecasting, deterministic and machine learning 
algorithms were more effective for predicting derived demand for front and 
back deployment. 

To evaluate d3ploy's capability in \Cyclus, transition scenarios 
are simulated using \Cyclus with and without it. 
The use of d3ploy automates the deployment of supporting facilities and thus, 
improves the ease of setting up transition scenarios. 
Users of the \Cyclus code can now run fuel cycle and transition scenarios simply 
by defining a driving commodity and its demand curve. 

%%%%%%%%%%%%%%%%%%%%%%%%%%%%%%%%%%%%%%%%%%%%%%%%%%%%%%%%%%%%%%%%%%%%%%%%%%%%%%%%
\bibliographystyle{ans}
\bibliography{bibliography}
\end{document}

