\documentclass{anstrans}
%%%%%%%%%%%%%%%%%%%%%%%%%%%%%%%%%%%
\title{Demonstration of Demand Driven Deployment Capabilities in \Cyclus}
\author{Gwendolyn J. Chee$^1$, Jin Whan Bae$^1$, Robert R. Flanagan$^2$, Roberto E. Fairhurst$^1$ and Kathryn D. Huff$^1$}

\institute{
$^1$Dept. of Nuclear, Plasma and Radiological Engineering, University of Illinois at Urbana-Champaign \\
$^2$Nuclear Engineering Program, University of South Carolina \\

gchee2@illinois.edu
}

%%%% packages and definitions (optional)
\usepackage{graphicx} % allows inclusion of graphics
\usepackage{booktabs} % nice rules (thick lines) for tables
\usepackage{microtype} % improves typography for PDF
\usepackage{xspace}
\usepackage{tabularx}
\usepackage{subcaption}
\usepackage{enumitem}
\usepackage{placeins}
\usepackage[acronym,toc]{glossaries}
\include{acros}
\makeglossaries
\newcommand{\SN}{S$_N$}
\renewcommand{\vec}[1]{\bm{#1}} %vector is bold italic
\newcommand{\vd}{\bm{\cdot}} % slightly bold vector dot
\newcommand{\grad}{\vec{\nabla}} % gradient
\newcommand{\ud}{\mathop{}\!\mathrm{d}} % upright derivative symbol
\newcommand{\Cyclus}{\textsc{Cyclus}\xspace}%
\newcommand{\Cycamore}{\textsc{Cycamore}\xspace}%
\newcolumntype{c}{>{\hsize=.56\hsize}X}
\newcolumntype{b}{>{\hsize=.7\hsize}X}
\newcolumntype{s}{>{\hsize=.74\hsize}X}
\newcolumntype{f}{>{\hsize=.1\hsize}X}
\newcolumntype{a}{>{\hsize=.45\hsize}X}
\usepackage{titlesec}
\titleformat*{\subsection}{\normalfont}

\begin{document}
%%%%%%%%%%%%%%%%%%%%%%%%%%%%%%%%%%%%%%%%%%%%%%%%%%%%%%%%%%%%%%%%%%%%%%%%%%%%%%%%
\section{Abstract}
\gls{NFC} simulation scenarios are constrained objective functions. 
The objectives are systemic demands such as "1\% power growth", while an 
example of a constraint is the availability of new nuclear technology. 
To add ease in setting up nuclear fuel cycle simulations, \gls{NFC} simulators 
should bring demand responsive deployment decisions into the dynamics of the 
simulation logic \cite{huff_current_2017}. 
While automated power production deployment is common in most fuel cycle 
simulators, automated deployment of supportive fuel cycle 
facilities is non-existent. 
Instead, the user must detail the deployment timeline of all supporting 
facilities or have infinite capacity support facilities. 
Thus, a next generation \gls{NFC} simulator should predictively and 
automatically deploy fuel cycle facilities to meet user defined power demand. 
This paper discusses demonstration of a new solution, \texttt{d3ploy}, enabling 
demand driven deployment in \Cyclus. 

\Cyclus is an agent-based nuclear fuel cycle simulation framework 
\cite{huff_fundamental_2016}. 
Each entity (i.e. Region, Institution, or Facility) in the fuel cycle is modeled 
as an agent. 
Institution agents
are responsible for deploying and decommissioning facility agents and
can represent a legal operating organization such as a 
utility, government, etc \cite{huff_fundamental_2016}. 

The Demand-Driven \Cycamore Archetypes project (NEUP-FY16-10512) aims to 
develop \Cyclus's demand-driven deployment capabilities. 
This capability is developed in the form of a \Cyclus Institution agent that 
deploys facilities to meet the front-end and back-end fuel cycle demands
based on a user-defined commodity demand. 
Its goal is to meet supply for any commodity while minimizing oversupply.
This demand-driven deployment capability is referred to as \texttt{d3ploy}. 

At each time step, \texttt{d3ploy} predicts demand and supply of each 
commodity for the following time step.
Then, \texttt{d3ploy} deploys facilities to 
meet predicted demand. 
Three interchangeable algorithm types govern demand and supply predictions: 
non-optimizing, time series forecasting, and deterministic optimizing. 

We compared the prediction algorithms with fuel cycle scenarios in which the
demand driving commodity, its demand curve, and the combination of facilities 
in the scenario are varied. 
Specifics about the prediction algorithms are discussed in 
\cite{flanagan_methods_2019}. 
We demonstrated deployment capabilities in \texttt{d3ploy} for scenarios 
requiring front-end facility deployment, back-end facility deployment, 
closed fuel cycle deployment, and transition scenario deployment.
Each scenario was run with each prediction algorithm, and the results 
were compared based on the number of time
steps in which demand exceeds supply and the $\chi^2$ goodness of fit test.
These comparisons demonstrated that the non-optimizing methods
effectively predicted appropriate facility deployment for supplying the 
driving commodity.  
These comparisons also demonstrated that the time series forecasting and 
deterministic algorithms effectively predicted appropriate 
facility deployment for supplying front-end and back-end commodities. 
The full paper will compare these prediction algorithms quantitatively, 
determine the advantages and disadvantages of each, and provide recommendations 
for which algorithm is desirable for use in modeling transition scenarios. 
Demonstration of how \texttt{d3ploy} improves the analysis of transition scenarios 
is discussed in \cite{fairhurst_implementation_2019}. 


%%%%%%%%%%%%%%%%%%%%%%%%%%%%%%%%%%%%%%%%%%%%%%%%%%%%%%%%%%%%%%%%%%%%%%%%%%%%%%%%
\bibliographystyle{ans}
\bibliography{bibliography}
\end{document}

